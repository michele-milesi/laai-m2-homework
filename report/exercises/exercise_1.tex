\section*{Exercise 1}

\subsection*{Exercise 1.4}
Observe that our model of evaluation allows for combinations whose operators are compound expressions. Use this observation to 
describe the behavior of the following procedure:

\begin{lstlisting}
    (define (a-plus-abs-b a b) 
        ((if (> b 0) + -) a b))
\end{lstlisting}

\paragraph{Solution}


\subsection*{Exercise 1.5}
Ben Bitdiddle has invented a test to determine whether the interpreter he is faced with is using applicative-order evaluation 
or normal-order evaluation. He defines the following two procedures:

\begin{lstlisting}
    (define (p) (p))

    (define (test x y)
        (if (= x 0) 
            0 
            y))
\end{lstlisting}

Then he evaluates the expression

\begin{lstlisting}
    (test 0 (p))
\end{lstlisting}

What behavior will Ben observe with an interpreter that uses applicative-order evaluation? What behavior will he
observe with an interpreter that uses normal-order evaluation? Explain your answer. (Assume that the evaluation
rule for the special form \texttt{if} is the same whether the interpreter is using normal or applicative order: the predicate 
expression is evaluated first, and the result determines whether to evaluate the consequent or the alternative expression).

\paragraph{Solution}