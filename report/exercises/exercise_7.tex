\section*{Exercise 7}

\begin{definition}[MC, functional view] 
    \normalfont A MC over a state space $X$ is a function $ c: X \to \mathcal{D}(X) $.
\end{definition}
\begin{definition}
    \normalfont A \textit{stochastic matrix} of dimension $n$ is a $ n \times n$-matrix $P$ whose entries belong to $[0, 1]$ and
    such that each row vector gives a distribution, i.e.:
    \[ \forall i.\; \sum_{j} P(i, j) = 1 \]
\end{definition}

\subsection*{Exercise 7 (1)}
\begin{proposition}
    The functional and matrix-based definitions of a MC are equivalent.
\end{proposition}

\begin{proof}[Proof sketch]
    Given $ c : X \to  D(X) $, with $ X = \{x_{1}, . . . , x_{n}\} $, we construct the 
    matrix $ P_{c} $ as $ P_{c}(i, j) = c(x_{i})(x_{j}) $. 
    Vice versa, given $ P $, we define $ c_{P}(x_{i})(x_{j}) = P(i, j) $.
\end{proof}

\paragraph{Instructions} \textit{Complete the above proof. Prove, in particular that for any $ x \in X $, $ c_{P} (x) $ is indeed a 
distribution; that $ c_{P} $ is a stochastic matrix; and that $ P_{c_{P}} = P $ and $ c_{P_{c}} = c $.}

\paragraph{Solution}
\begin{proof}
    To prove that the matrix $P_{c}$ is a stochastic matrix, we have to prove that the sum of the elements of each row of the matrix
    $P_{c}$ is equal to $1$.
    Since we have defined the matrix $P_{c}$ as $ P_{c}(i, j) = c(x_{i})(x_{j}) $, we can write the follwing equality:
    \[ \forall i\;\sum_{j} P_{c}(i, j) = \sum_{j} c(x_{i})(x_{j}) = 1 \]
    because $c$ is a Markov Chain over the state space $X$.

   On the other hand, we need to prove that $c_{P}$ is a distribution. We can procede as before: we have defined $c_{P}$ as
   $ c_{P}(x_{i})(x_{j}) = P(i, j)$, thus it is possible to write the follwing equality:
   \[ \forall i\;\sum_{j} c_{P}(x_{i})(x_{j}) = \sum_{j} P(i, j) = 1 \]
   because $P$ is a stochastic matrix, so each row of the matrix gives a distribution, thus the sum of the elements of each row is
   equal to $1$.
\end{proof}


\subsection*{Exercise 7 (2)}
\textit{Prove that $ c(x) = c^{*}(\delta_{x}) $.}

\paragraph{Solution}


\subsection*{Exercise 7 (3)}
\textit{Prove that $ c^{*}(\psi) = \psi(P_{c}) $.}

\paragraph{Solution}


\subsection*{Exercise 7 (4)}
\textit{Prove that f $ \psi $ satisfied DBC, then $ \psi $ is stationary for $ P $.}

\paragraph{Solution}