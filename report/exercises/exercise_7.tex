\section*{Exercise 7}

\begin{definition}[MC, functional view] 
    \normalfont A MC over a state space $X$ is a function $ c: X \to \mathcal{D}(X) $.
\end{definition}
\begin{definition}
    \normalfont A \textit{stochastic matrix} of dimension $n$ is a $ n \times n$-matrix $P$ whose entries belong to $[0, 1]$ and
    such that each row vector gives a distribution, i.e.:
    \[ \forall i.\; \sum_{j} P(i, j) = 1 \]
\end{definition}

\subsection*{Exercise 7 (1)}
\begin{proposition}
    The functional and matrix-based definitions of a MC are equivalent.
\end{proposition}

\begin{proof}[Proof sketch]
    Given $ c : X \to  D(X) $, with $ X = \{x_{1}, . . . , x_{n}\} $, we construct the 
    matrix $ P_{c} $ as $ P_{c}(i, j) = c(x_{i})(x_{j}) $. 
    Vice versa, given $ P $, we define $ c_{P}(x_{i})(x_{j}) = P(i, j) $.
\end{proof}

\paragraph{Instructions} \textit{Complete the above proof. Prove, in particular that for any $ x \in X $, $ c_{P} (x) $ is indeed a 
distribution; that $ c_{P} $ is a stochastic matrix; and that $ P_{c_{P}} = P $ and $ c_{P_{c}} = c $.}

\paragraph{Solution}
\begin{proof}
    To prove that the matrix $P_{c}$ is a stochastic matrix, we have to prove that the sum of the elements of each row of the matrix
    $P_{c}$ is equal to $1$.
    Since we have defined the matrix $P_{c}$ as $ P_{c}(i, j) = c(x_{i})(x_{j}) $, we can write the follwing equality:
    \[ \forall i.\;\sum_{j} P_{c}(i, j) = \sum_{j} c(x_{i})(x_{j}) = 1 \]
    because $c$ is a Markov Chain over the state space $X$.

   On the other hand, we need to prove that $c_{P}$ is a distribution. We can procede as before: we have defined $c_{P}$ as
   $ c_{P}(x_{i})(x_{j}) = P(i, j)$, thus it is possible to write the follwing equality:
   \[ \forall i.\;\sum_{j} c_{P}(x_{i})(x_{j}) = \sum_{j} P(i, j) = 1 \]
   because $P$ is a stochastic matrix, so each row of the matrix gives a distribution, thus the sum of the elements of each row is
   equal to $1$.
\end{proof}


\subsection*{Exercise 7 (2)}
\textit{Prove that $ c(x) = c^{*}(\delta_{x}) $.}

\paragraph{Solution}
The Dirac distribution of a state $x_{k}$ is defined as follows:
\[ \delta_{x_{k}} = 
    \begin{cases}
        x_{k} = 1 \\
        x_{i} = 0 & \text{if } i \neq k
    \end{cases}
\]
We have defined the map $c^{*}(\phi)(y) = \sum_{x} \phi(x) \cdot c(x)(y)$, so we can write:
\[ \forall y.\;c^{*}(\delta_{x_{k}})(y) = \sum_{x_{i}} \delta_{x_{k}}(x_{i}) \cdot c(x_{i})(y) = 1 \cdot c(x_{k})(y) = c(x_{k})(y) \]
Since the Dirac distribution is always zero except for $x_{k}$ that is equal to $1$.
This equality holds for each state $y$, so have proved that:
\[ c^{*}(\delta_{x_{k}}) = c(x_{k}) \]

The notation is slightly different from the one of the instructions, since I do not want to create confusion when the summation is
written. Indeed, the goal is to make clear that the summation is defined over all the states of the \textit{state space} and that
the only addend that is not null is the one for which the value of $\delta_{x_{x}}(x_{i}) = 1$ (i.e. for $x_{k}$).


\subsection*{Exercise 7 (3)}
\textit{Prove that $ c^{*}(\psi) = \psi(P_{c}) $.}

\paragraph{Solution}
The product $\psi P_{c}$ returns a vector of dimension $(1, n)$ where $n$ is the number of states. 
Furthermore it is important to remember that: \textit{(i)} The matrix $ P_{c} $ has been defined as $ P_{c}(i, j) = c(x_{i})(x_{j}) $;
\textit{(ii)} The map $c^{*}$ has been defined as follows: $ c^{*}(\phi)(y) = \sum_{x} \phi(x) c(x)(y) $.
Thus we can write the following equality:
\[ \forall y.\;(\psi P_{c})(y) = \sum_{x} \psi(x) \cdot P_{c}(x, y) \forone \sum_{x} \psi(x) c(x)(y) \fortwo c^{*}(\psi)(y) \]
Since this holds forall $y$, then we can assert that $c^{*}(\psi) = \psi P_{c}$.


\subsection*{Exercise 7 (4)}
\textit{Prove that if $ \psi $ satisfied DBC, then $ \psi $ is stationary for $ P $.}

\paragraph{Solution}
Before starting with the demonstration, it is important to remember the definition of \textit{DBC}.
\begin{definition}
    \normalfont Given a MC $P$, we say that a distribution $\psi$ satisfies \textit{detailed balanced condition} (DBC) if
    \[ \forall x, y.\;\psi(x) \cdot P(x, y) = \psi(y) \cdot P(y, x) \]
\end{definition} 
Since $\psi$ satisfies DBC, the equality above holds for each state $x, y$. Thus it is possible to write the following equation:
\[ 
    \forall y.\;(\psi P)(y) = \sum_{x} \psi(x) \cdot P(x, y)\;\;\;\fordbc\;\;\;\sum_{x} \psi(y) \cdot P(y, x) = 
    \psi(y) \cdot \sum_{x} P(y, x) = \psi(y)
\]
The last equality is due to the fact that is a stochastic matrix, so the sum of the elements of a row is equal to $1$.
Since the equation holds for each $y$, then it is possible to assert that $\psi P = \psi$.