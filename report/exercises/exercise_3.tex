\section*{Exercise 3}

\subsection*{Exercise 3.2}
Explain why (in terms of the evaluation process) these two programs give different answers 
(i.e. have different distributions on return values):
\begin{lstlisting}
    (define foo (flip))
    (list foo foo foo)
\end{lstlisting}

\begin{lstlisting}
    (define (foo) (flip))
    (list (foo) (foo) (foo))
\end{lstlisting}    

\paragraph{Solution}


\subsection*{Exercise 3.5}
Here is a modified version of the tug of war game. Instead of drawing strength from the continuous Gaussian 
distribution, strength is either 5 or 10 with equal probability. Also the probability of laziness is changed from 1/4 to 1/3. 
Here are four expressions you could evaluate using this modified model:

\begin{lstlisting}
    (define strength (mem (lambda (person) (if (flip) 5 10))))
    (define lazy (lambda (person) (flip (/ 1 3))))

    (define (total-pulling team)
      (sum
       (map (lambda (person)
              (if (lazy person)
                  (/ (strength person) 2) 
                  (strength person)))
            team)))

    (define (winner team1 team2) 
      (if (< (total-pulling team1) (total-pulling team2)) 
          team2 
          team1))

    (winner '(alice) '(bob))                        ;; expression 1

    (equal? '(alice) (winner '(alice) '(bob)))      ;; expression 2

    (and (equal? '(alice) (winner '(alice) '(bob))) ;; expression 3
         (equal? '(alice) (winner '(alice) '(fred))))

    (and (equal? '(alice) (winner '(alice) '(bob))) ;; expression 4
         (equal? '(jane) (winner '(jane) '(fred))))
\end{lstlisting}

\begin{itemize}
    \item[a.] Write down the sequence of expression evaluations and random choices that will be made in evaluating each expression.
    \item[b.] Directly compute the probability for each possible return value from each expression.
    \item[c.] Why are the probabilities different for the last two? Explain both in terms of the probability calculations 
        you did and in terms of the “causal” process of evaluating and making random choices.
\end{itemize}

\paragraph{Solution}


\subsection*{Exercise 3.6}
Use the rules of probability, described above, to compute the probability that the geometric distribution 
defined by the following stochastic recursion returns the number 5.

\begin{lstlisting}
    (define (geometric p)
      (if (flip p)
          0
          (+ 1 (geometric p))))
\end{lstlisting}

\paragraph{Solution}


\subsection*{Exercise 3.7}
Convert the following probability table to a compact Church program:
\begin{table}[h]
    \begin{center}
        \begin{tabular}{ccc}
            \hline
            A & B & P(A, B) \\
            \hline
            F & F & 0.14 \\
            F & T & 0.06 \\
            T & F & 0.4 \\
            T & T & 0.4 \\
            \hline
        \end{tabular}
    \end{center}
    
\end{table}

Hint: fix the probability of A and then define the probability of B to depend on whether A is true or not. 
Run your Church program and build a histogram to check that you get the correct distribution.

\begin{lstlisting}
    (define a ...)
    (define b ...)
    (list a b)
\end{lstlisting}

\paragraph{Solution}